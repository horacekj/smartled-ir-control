\documentclass[12pt,a4paper]{article}
\usepackage[czech]{babel}
\usepackage[utf8]{inputenc}
\usepackage[T1]{fontenc}
\usepackage{lmodern}
\usepackage{microtype}
\textwidth 16cm \textheight 23.6cm
\topmargin -1.3cm
\oddsidemargin 0cm
\usepackage{url}
\usepackage{listings}
\usepackage{paralist}
\usepackage[ pdfauthor={Jan Horacek}
           , unicode
           , draft=false
           , colorlinks=true
           , unicode=true
           ]{hyperref}

\begin{document}

\setcounter{page}{1}  % nastaví čítač stránek znovu od jedné
\pagenumbering{arabic} % číslování arabskymi číslicemi

\title{\normalsize Projekt předmětu PB171 \\ \huge Dálkové ovládání SmartLED pásku
\\ \normalsize semestr jaro 2019, FI MUNI}
\author{Jan Horáček\\ 445326}
\date{\today}
\maketitle

\setlength{\parindent}{0cm}
\setlength{\parskip}{3mm plus2pt minus2pt}

%%%%%%%%%%%%%%%%%%%%%%%%%%%%%%%%%%%%%%%%%%%%%%%%%%%%%%%%%%%
\section{Zadání}

Vytvořte systém umožňující ovládání SmartLED pásku pomocí standardního IR
dálkového ovladače. Pochopte a implementujte komunikační protokoly SmartLED
pásků a IR dálkového ovladače.

%%%%%%%%%%%%%%%%%%%%%%%%%%%%%%%%%%%%%%%%%%%%%%%%%%%%%%%%%%%
\section{Popis řešení}

Vypracované řešení staví na desce \textit{YuniPIC}, kterou jsme využívali
v~předmětu PB171 během semestru. Řešení nevyužívá žádných specialit desky,
tato deska byla zvolena pro její snadnou dostupnost. Řešení je psáno na míru
procesoru PIC18F46K22. Projekt využívá asi nejrozšířenější chytré LED diody
\textit{WS2812B}, jako IR protokol byl použit rozšířený \textit{NEC IR}.

%%%%%%%%%%%%%%%%%%%%%%%%%%%%%%%%%%%%%%%%%%%%%%%%%%%%%%%%%%%
\section{SmartLED protokol}

Chytré LE diody \textit{WS2812B} komunikují jednoduchým sériovým protokolem.
LED pásek je připojen pouze třemi vodiči: napájení, zem a data. Každá LE dioda
z~příchozivších dat usekne 24 bitů, podle kterých nastaví svou barvu, a zbylé
bity pošle dát dalším LED. Po delší pauze ($> 50\ us$) se stav LED resetuje,
LED tedy další příchozivší data nepřepošle, ale vezme pro sebe.

Jednička je kódována jako $0.4\ us$ nízké úrovně a $0.85\ us$ vysoké úrovně,
nula je kódována jako $0.8\ us$ nízké úrovně a $0.45\ us$ vysoké úrovně.
\footnote{\url{https://cdn-shop.adafruit.com/datasheets/WS2812B.pdf}}

%%%%%%%%%%%%%%%%%%%%%%%%%%%%%%%%%%%%%%%%%%%%%%%%%%%%%%%%%%%
\section{IR protokol}

Protože IR signálu je kolem nás spousta, téměř všechny bezdrátové IR ovladače
využívají modulace dat na nosné frekvenci $37.9\ kHz$. Tento princip využívá
i tento projekt. K~procesoru je připojen přijímač \textit{OSRB38C9BA}
\footnote{\url{https://www.gme.cz/osrb38c9ba}}, který rovnou řeší demodulaci
a filtraci, takže do procesoru jde již krásný obdélníkový signál.

Všichni výrobci IR ovladačů sice modulují konzistentně, avšak formát modulovaných
dat se již mezi výrobci ovladačů liší. Tento projekt využívá rozšířený formát
\textit{NEC IR}\footnote{\url{https://exploreembedded.com/wiki/NEC_IR_Remote_Control_Interface_with_8051}}.
Bity jsou kódovány pomocí délek pulsů, měří se vždy od náběžné k~náběžné hraně,
na náběžné hrany je tedy možno synchronizovat. Protokol přenáší

\begin{compactenum}
\item start bit,
\item adresu ovladače,
\item logickou negaci adresy ovladače (pro kontrolu),
\item kód příkazu,
\item logickou inverzi kódu příkazu (pro kontrolu).
\end{compactenum}

%%%%%%%%%%%%%%%%%%%%%%%%%%%%%%%%%%%%%%%%%%%%%%%%%%%%%%%%%%%
\section{Zapojení hardware}

IR přijímač je připojen na napájecí napětí desky $+5\ V$, datový vodič je
připojen na pin \textit{RA0} (aby bylo možné využít \textit{capture režim}
procesoru).  Pro správný příjem signálu je nutné odstranit kondenzátor
\textit{C12}, který je k~tomuto pinu přidružen.

LED pásek je připojen na pin \textit{RA0} (může být připojen na libovolný
výstupní pin, tento byl zvolen pro snadnou dostupnost).

Program dále využívá LED na desce pro signalizaci stavu, žádný další hardware
není využit.

%%%%%%%%%%%%%%%%%%%%%%%%%%%%%%%%%%%%%%%%%%%%%%%%%%%%%%%%%%%
\section{Popis činnosti}

Program očekává $46$ diod na LED pásku. Po startu všechny LED zhasne a vyčkává
na stisk tlačítka ovládače. Na ovladači je možné zvolit jeden z~předpřipravených
módu:

\begin{compactenum}
\item Svítí konkrétní barva (červená, zelená, žlutá, modrá, bílá).
\item Střídají se barvy \texttt{RGBRGB...} a v~čase se posouvají.
\item Všechny LED plynule mění barvu dle barvy duhy (v~čase).
\item Mód \textit{Knight Rider}: po pásku běhá červená barva.
\end{compactenum}

Tlačítky \texttt{Ch+} a \texttt{Ch-} je možné ovládat intenzit barev.

%%%%%%%%%%%%%%%%%%%%%%%%%%%%%%%%%%%%%%%%%%%%%%%%%%%%%%%%%%%
\section{Implementace}

Firmware je psaný modulárně, každý soubor se stará vždy o~jednu věc:
\begin{compactitem}
	\item \texttt{smartled.h}, \texttt{smartled.c}: nastavování LED pásku,
	\item \texttt{ir.h}, \texttt{ir.c}: příjem a parsování IR komunikace,
	\item \texttt{main.c}: spojení modulů, módy pásku.
\end{compactitem}

Největším problémem v~celém FW byla komunikace s~LED páskem: protokol, kterým
LED komunikují, je totiž velice rychlý. PIC nemá žádnou nativní podporu pro
tento protokol, je tedy nutné data generovat manuálně bit po bitu. To znamená
rozlišovat mezi intervaly v~řádu desítek pikosekund a to vše na frekvenci $16\
MHz$ interního oscilátoru. Firmware byl upraven tedy tak, aby plně využíval
\textit{PLL} procesoru, frekvence tedy byla navýšena na $64\ MHz$. Časy
protokolu jsou tak krátké, že použití časovače nepřichází v~úvahu. Signál je
tedy generován jedním blokujícím voláním funkce, které provádí čekání pomocí
operace \texttt{Nop}.  Navíc bylo nutné ručně \textit{unrollovat} cyklus
odesílající oněch 24 bitů pro každou LED, protože režie cyklu byla tak velká,
že nebylo možné dodržet časování protokolu. Mezi jednotlivými LED je větší
pauza, než protokol dovoluje, avšak zdá se, že s~tím LED nemají problém.

Příjem IR signálu je postaven na \textit{capture feature} procesoru, která
umožňuje přesně měřit periodu vstupního signálu. Pro toto měření je použit
časovač 1.

Zbytek kódu je poměrně přímočarý a bez zásadních záludností.

%%%%%%%%%%%%%%%%%%%%%%%%%%%%%%%%%%%%%%%%%%%%%%%%%%%%%%%%%%%
\section{Diskuze}

Překvapilo mě, jak velký boj byl s~generováním signálu pro \textit{SmartLED}.
Díval jsem se do implementace knihovny
\textit{FastLED}\footnote{\url{https://github.com/FastLED/FastLED}}, která
implementuje komunikaci právě s~těmito LED na platformách, jako třeba Arduino.
Po projití části kódu jsem k~této knihovně získal respekt. Zjistil jsem, že
častým trikem pro generování signálu je využít SPI (data tohoto formátu na něm
jsou vytvořit) kombinovaný s~\textit{DMA}. Bohužel, \textit{DMA} není na
procesorech rodiny \textit{PIC18F46} k~dispozici (umožňují jej až 32-bitové
PICy). Proto jsem tento přístup ani nezkoušel.

Přesně pro takovéto signály se hodí moderní periferie procesorů, například
ESP32 má přímo hardwarový generátor, který by tento typ signálu zvládl hravě
vytvořit.

\end{document}
